%!TEX root = ../dissertation.tex

\chapter{Research Results}
\label{chapter:results}
%
In this section, we are going to dive into the results obtained from the user study described on chapter number \ref{chapter:design}.
On the first section, we will clearly explain the test protocol which was followed by the users in the laboratory environment to correctly
execute the study; this section will be followed, not only by the description of how the gathered data was treated and cleaned
(Section \ref{sec:results_datacleaning}), but also the transformation of this data using \emph{Matlab} processing tools, in order to prepare
it for the statistical scrutiny (Section \ref{sec:results_digest}). Hereafter, conclusions will be drawn from the study at section
\ref{sec:results_results}, when trying to find answers to the questions/objectives raised before. \par
%
The final section of this chapter will be dedicated to summarize the results and infer important conclusions, implications and guidelines which
could be relevant for the InfoVis field of research.
%
\section{Protocol}
\label{sec:results_protocol}
%
The existence of a test protocol, when performing a User Study is mandatory: without it, the test may not follow a strictly previously defined
standard. As written before, this user study was conducted two-pronged: in a laboratory environment and \emph{via} online dissemination channels. \par
%
The users were given always the same briefing when they arrived at the user study test site: it was explained the motivation behind the master
thesis, the goals which were expected for this phase of the study and what was expected for them to execute. The most important information which
was told was that \ul{"there was no pre-defined correct and wrong answers to each question, this test was designed to test the general
color mixing capabilities of the majority of the users"}. Besides this information, the user study was self-contained, in the sense that
every other relevant information and instruction was in the interface, adapted for each test phase, so it was not given any physical artifacts
describing instructions. The instructions were available on two languages, depending on the choice of the user: Portuguese and English. \par
%
Before each session of the laboratory environment test-run, a Datacolor Spyder 5 Elite Color Calibrator was USB-connected to the computer and, using
the software which is shipped along with it, the computer LCD display was fully calibrated (the software offers the option of recalibrating,
the option of checking the calibration and also, the option of fully calibrating the display) by testing the pixel emission when emmiting a particular
set of colors. The display was everytime fully calibrated, since the software manisfested an erratic bahaviour when using the other functions:
the screen colors were presented in a very warmer/colder color profile tha it was before. \par
%
The tests were conducted at, most of the repetitions, in \gls{RNL} at \gls{IST}, and fewers times in other locations with similar conditions:
this is due to constraints in finding users, so the test site needed to have a (limited) mobility feature. However, the conditions remained
the same concerning the illumination, the position of the user and the computer used: a Macbook Air 13' (Mid-2013) was prepared undeneath
a fixed incadescent light-source (but slightly deviated from it, to minimize light reflections on screen), the user would sit in front
of the laptop, in an almost silent environment. Ideal conditions of this test would be such that the user could be sitting alone in a completely
silent room, his head would be always at the same distance from the screen, resting in a head-rest and the LCD display's inclination would be perfectly
adjusted to the user's eyes. \par
%
On the online environment, as easily predictable, cannot be completly controlled. For the sake of calibration, it was asked the user
to perform a set of six calibration easy steps before starting the test, so the online user's screen would be,
somehow, in a standardized calibration fashion. The calibration steps which were asked are:
%
\begin{enumerate}
  \item If possible, adjust your room lights for a comfortable usage of your device.
  \item Avoid reflections on your screen, by diverting the screen from direct sources of light. This step is important,
  since light reflections can affect visualization of images.
  \item To adjust the \textbf{Black Point} of your screen, define the \ul{Contrast} and \ul{Brightness} of your screen to their maximum.
  \item After Step 3, gradually reduce \textbf{Brightness} value of your screen, in order to correctly distinguish the squares of each image below [calibration squares images].
  \item If possible, define the \textbf{Color Temperature} of your screen to 6500 Kelvin Degrees.
  \item You are now ready to answer the following questions!
\end{enumerate} \par
%
The ideal conditions of this test would be such that we could control and maniupulate the color calibration of online user's LCD display, using a software
piece which would acquire important informations from the screen configuration, \emph{e.g.} resolution, white-point, black-point, brightness, among others,
digest the values and present the questions from the Core Phase in a completely controled and calibrated window. Further investigation could focus in
developing this system. \par
%
The users were asked to fill in a profiling questionnaire (as seeen of section \ref{subsec:design_profiling}), as well as to respond to calibration form
(Section \ref{subsec:design_calibration}). A validated simplified 6-plate Ishihara color blindness test \cite{Alwis1992} is, then performed
(Section \ref{subsec:design_ishihara}), before proceeding onto the 32 questions test-phase, in which the user is asked to slide one(two) circular
object(s) placed on top of a bar, to indicate a(the) color(s) which he thought were the correct mixture answers. In the end, the user could leave
feedback, by sending a message which would be stored in a Relational Database. \par
%
The instructions which were presented in each page can be consulted in Appendix \ref{appendix:protocol}. \par
%
\section{Data Cleaning}
\label{sec:results_datacleaning}
%
Referir problema que foi detectado entre Chrome e Safari, cores de calibração.
%
\section{Data Processing}
\label{sec:results_digest}
%
Como foram tratados os dados, no Matlab? Como foram preparados para a análise?
Que processamento for feito aos dados? \\
Incluir a mesma tabela que em \ref{subsec:design_core}, com cores, mas com numero de respostas online, lab, e demo.
%
\subsection{Data Preparation}
\label{subsec:results_preparation}
%
Conversão de respostas com perfil icc novamente, mistura das respostas de acordo com modelos de cor, etc.
Detalhar.
%
\subsection{Color Bins Comparation}
\label{subsec:results_preparation}
%
De onde apareceram os Bins, como eles são em bruto, incluir esquema do XKCD. Que tratamento foi dado, os problemas
com o desenho dos mesmos e a comparação contra os pontos (em vez da área, que seria o ideal). O que esperavamos
(áreas bem definidas, poligonos bem delineados que daria para desenhar o convexhull), as colisões entre áreas (valores
comuns entre alguns Color Bins) e falar do facto de como se podia, alternativamente, encontrar o nome das cores (diagrama
que já existe desde 1974 - ver ref - mas que não existe um svg ou codigo de todos os pontos, pelo que ainda havia essa curva
de implementação). Poder-se-ia atribuir um nome à cor pela temperatura da mesma, mas não existia uma tabela credível de valores
a utilizar.
%
\section{Results}
\label{sec:results_results}
%
Interpretação de valores. Começar por valores de laboratório. Online serve para corroborar. \par
Incluir tantas tabelas quantas necessárias em cada secção, adaptadas a cada secção e não standardizadas.
%
\subsection{User Profile}
\label{subsec:results_userprofile}
%
\subsection{Color Mixtures}
\label{subsec:results_colormixtures}
%
Fazer também mistura mais fácil, comparando os ratings das questões e ver qual a mistura que apresenta melhores resultados. \\
Comparar misturas que originam a mesma cor, com base em primárias diferentes e perceber se utilizadores conseguem detectar várias
misturas para uma mesma cor.
%
\subsection{Color Models}
\label{subsec:results_colormodels}
%
\subsection{Color Naming}
\label{subsec:results_namingcolors}
%
Cores mais comuns em algumas perguntas; existe alguma ordem caracteristica quando utilizador especifica uma mistura?
%
\subsection{Demographic Groups}
\label{subsec:results_demographic}

\section{Discussion}
\label{sec:results_discussion}

\subsection{Calibration Resiliency}
\label{subsec:results_calibration}
%
Como verificamos ainda alguns users com calibração imprópria para teste, considerámos que poderia ser uma fonte de resultados
interessantes. Como tal, criámos um dataset para os mesmos e comparámos com os resultados dos utilizadores calibrados. Os resultados
são os que se seguem... \par
%
\subsection{Creation of Color Scales}
\label{subsec:results_discussion_colorscales}

\subsection{Color Organization}
\label{subsec:results_discussion_colororganization}

\subsection{Consequences for InfoVis}
\label{subsec:results_discussion_infovis}
%
Resumo dos resultados todos e regras que se podem levar deste trabalho para a área de InfoVis em geral.
%
