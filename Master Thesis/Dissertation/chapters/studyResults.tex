%!TEX root = ../dissertation.tex

\chapter{Color Mixing Perception: Analyzing Results}
\label{chapter:results}
%
In this section, we are going to dive into the results obtained from the user study described on chapter number \ref{chapter:design}.
On the first section, we will clearly explain the test protocol which was followed by the users in the laboratory environment to correctly
execute the study; this section will be followed, not only by the description of how the gathered data was treated and cleaned
(Section \ref{sec:results_datacleaning}), but also the transformation of this data using \emph{Matlab} processing tools, in order to prepare
it for the statistical scrutiny (Section \ref{sec:results_digest}). Hereafter, conclusions will be drawn from the study at section
\ref{sec:results_results}, when trying to find answers to the questions/objectives raised before. \par
%
The final section of this chapter will be dedicated to summarize the results and infer important conclusions, implications and guidelines which
could be relevant for the InfoVis field of research.
%
\section{Protocol}
\label{sec:results_protocol}
%
The existence of a test protocol, when performing a User Study is mandatory: without it, the test may not follow a strictly previously defined
standard. As written before, this user study was conducted two-pronged: in a laboratory environment and \emph{via} online dissemination channels. \par
%
\subsection{Laboratory Environment}
%
The users were given always the same briefing when they arrived at the user study test site: it was explained the motivation behind the master
thesis, the goals which were expected for this phase of the study and what was expected for them to execute. The most important information which
was told was that \ul{"there was no pre-defined correct and wrong answers to each question, this test was designed to test the general
color mixing capabilities of the majority of the users"}. Besides this information, the user study was self-contained, in the sense that
every other relevant information and instruction was in the interface, adapted for each test phase, so it was not given any physical artifacts
describing instructions. The instructions were available on two languages, depending on the choice of the user: Portuguese and English. \par
%
Before each session of the laboratory environment test-run, a Datacolor Spyder 5 Elite Color Calibrator was USB-connected to the computer and, using
the software which is shipped along with it, the computer LCD display was fully calibrated (the software offers the option of recalibrating,
the option of checking the calibration and also, the option of fully calibrating the display) by testing the pixel emission when emmiting a particular
set of colors. The display was everytime fully calibrated, since the software manisfested an erratic bahaviour when using the other functions:
the screen colors were presented in a very warmer/colder color profile tha it was before. \par
%
The tests were conducted at, most of the repetitions, in \gls{RNL} at \gls{IST}, and fewers times in other locations with similar conditions:
this is due to constraints in finding users, so the test site needed to have a (limited) mobility feature. However, the conditions remained
the same concerning the illumination, the position of the user and the computer used: a Macbook Air 13' (Mid-2013) was prepared undeneath
a fixed incadescent light-source (but slightly deviated from it, to minimize light reflections on screen), the user would sit in front
of the laptop, in an almost silent environment. Ideal conditions of this test would be such that the user could be sitting alone in a completely
silent room, his head would be always at the same distance from the screen, resting in a head-rest and the LCD display's inclination would be perfectly
adjusted to the user's eyes. \par
%
\subsection{Online Environment}
%
Performing the study online, as easily predictable, develops some characteristics which cannot be completly controlled. For the sake of calibration, it was asked the user
to perform a set of six calibration easy steps before starting the test, so the online user's screen would be,
somehow, in a standardized calibration fashion. The calibration steps which were asked are:
%
\begin{enumerate}
  \item If possible, adjust your room lights for a comfortable usage of your device.
  \item Avoid reflections on your screen, by diverting the screen from direct sources of light. This step is important,
  since light reflections can affect visualization of images.
  \item To adjust the \textbf{Black Point} of your screen, define the \ul{Contrast} and \ul{Brightness} of your screen to their maximum.
  \item After Step 3, gradually reduce \textbf{Brightness} value of your screen, in order to correctly distinguish the squares of each image below [calibration squares images].
  \item If possible, define the \textbf{Color Temperature} of your screen to 6500 Kelvin Degrees.
  \item You are now ready to answer the following questions!
\end{enumerate} \par
%
The ideal conditions of this test would be such that we could control and maniupulate the color calibration of online user's LCD display, using a software
piece which would acquire important informations from the screen configuration, \emph{e.g.} resolution, white-point, black-point, brightness, among others,
digest the values and present the questions from the Core Phase in a completely controled and calibrated window. Further investigation could focus in
developing this system. \par
%
The users were asked to fill in a profiling questionnaire (as seeen of section \ref{subsec:design_profiling}), as well as to respond to calibration form
(Section \ref{subsec:design_calibration}). A validated simplified 6-plate Ishihara color blindness test \cite{Alwis1992} is, then performed
(Section \ref{subsec:design_ishihara}), before proceeding onto the 32 questions test-phase, in which the user is asked to slide one(two) circular
object(s) placed on top of a bar, to indicate a(the) color(s) which he thought were the correct mixture answers. In the end, the user could leave
feedback, by sending a message which would be stored in a Relational Database. \par
%
The instructions which were presented in each page can be consulted in Appendix \ref{appendix:protocol}. \par
%
\section{Data Cleaning}
\label{sec:results_datacleaning}
%
Throughout the user study, we collected a \textbf{total amount of four-hundred and seventy-nine (479) users} which interacted with our study and fulfilled,
at least, until the Color Vision Deficiencies Test Phase. However, only \textbf{two-hundred and sixty-one (261) users went on to the core phase of the study},
representing \textbf{54.48\%} of the total amount, giving at least one answer on the set of 32 questions, loosing the other two hundred and eighteen
(218) users which did not leave any answer, defining the remaining percentage \textbf{45.52\%}. This large drop of users could be due to errors
reported by the users, apparently the inability to submit answers when using the "Submit" button after rating the question; there were also some complaints
when users tried to perform the study in some mobile devices (namely, the \emph{iPhone\textsuperscript{\textregistered} 6}), whereon the color slider was
not able to be dragged and change the color value at the user will. \par
%
Concerning the percentage of users which showed up at the \textbf{laboratory trials, there were twenty-nine (29) users who performed the entire study}.
On the other hand, there were \textbf{two-hundred and thirty-two (232) users which carried out the study online}. There was also a small sample of color vision
deficient users which we will analyze in a qualitative manner; this set of users contains only one (1) user from the Laboratory Environment - \textbf{3.44\%} of
the sample size - and two (2) online users - \textbf{less than 1\%}. Lastly, we detected a small percentage of \textbf{six (6) users (2.59\%) which did not
presented a correct calibration of its LCD display}, evaluation based on the criteria referred before.
%
The data presented and used in this dissertation document was gathered along roughly two months, from 15th of April until 8th of June. As said before,
it was collected both with online and laboratory users, which was therefore stored in a Relational Database as previously explained in section
\ref{sec:impl_designingsolution}. \par
%
In the end of the study, \gls{CSV} files were exported from each table using a PostgreSQL for macOS called
\emph{Postico}\footnote{Postico - a modern PostgreSQL client for the Mac, Available at: \url{eggerapps.at/postico/}. Last accessed on
September 11th, 2016.}, which originated five files containing raw data to be cleaned and processed. The files were as follows:
%
\begin{itemize}
  \item \emph{raw\_data\_user\_profile.csv} - Data aggregated from "Profiling", "Calibration" and "Color Vision Deficiencies" Tables;
  \item \emph{raw\_data\_first\_profiling.csv} - Data from "Profiling" Table;
  \item \emph{raw\_data\_first\_calibration.csv} - Data from "Calibration" Table;
  \item \emph{raw\_data\_first\_ishihara.csv} - Data from "Color Vision Deficiencies" Table;
  \item \emph{raw\_data\_first\_results.csv} - Data from "Results" Table;
\end{itemize} \par
%
\begin{table}[htbp]
  \resizebox{\textwidth}{!} {
  \begin{tabular} {|c|c|c|c|c|c|c|c|c|c|}
    \hline
    User ID & Type & First Color & Second Color & Third Color & Drags & Time & Rating & Resets & Question ID \\ \hline
    5710cca334d60 & objTwoColors & \#0080FF & hsl(58.69565217391305,1,0.50) & hsl(98.15217391304348,1,0.50) & 992 & 117 & 4 & 2 & 10 \\ \hline
    5745c1c07cc0c & objTwoColors & \#8000FF & hsl(300,1,0.50) & hsl(324.13043478260875,1,0.50) & 645 & 55 & 2 & 1 & 14 \\ \hline
    5745350dc1e22 & objTwoColors & \#0080FF & hsl(226.30434782608697,1,0.50) & NONE & 115 & 11 & 5 & 1 & 10 \\ \hline
    57451c3b38192 & objTwoColors & \#00FF80 & NONE & hsl(150,1,0.50) & 462 & 39 & 5 & 1 & 15 \\ \hline
    574511e99b6d9 & objTwoColors & \#0080FF & hsl(15.652173913043478,1,0.50) & hsl(316.30434782608694,1,0.50) & 442 & 40, & 1 & 1 & 10 \\ \hline
    57427cf6bad0c & twoColorsObj & \#00FFFF & \#FFFF00 & \#46FF9C & 6 & 14 & 3 & 1 & 32 \\ \hline
    5740bda9be3dc & objTwoColors & \#FF7200 & hsl(9.130434782608695,1,0.50) & hsl(50.21739130434783,1,0.50) & 45 & 22 & 5 & 1 & 11 \\ \hline
    573c783748e8b & twoColorsObj & \#00FFFF & \#CBFF00 & \#00FF6B & 44 & 25 & 3 & 1 & 32 \\
    \hline
  \end{tabular}}
  \caption[Excerpt of Raw "Results" Table]{Excerpt of Results Table, with raw data.}
  \label{table:csv_resultsraw}
\end{table} \par
%
The refined tables were then divided into new and more specific ones so that we could detail our results analysis according to the goals defined before;
the "Results" table was refined into \ul{Laboratory Results}, \ul{Online Results} and demographic results: concerning the age, we divided it
on \ul{Users aged below 20 Years Results}, \ul{Users aged between 20 and 29 Years Results}, \ul{Users aged between 30 and 39 Years Results},
\ul{Users aged between 40 and 49 Years Results}, \ul{Users aged between 50 and 59 Years Results} and \ul{Users aged above 60 Years Results}.
Respecting the division of genders, we created the categories \ul{Female Users Results}, \ul{Male Users Results} and \ul{Other Gender Users Results}.
An excerpt of raw data contained in "Results" table can be found in table \ref{table:csv_resultsraw}; this allows us to support the explanation of the
following steps of the cleaning phase. \par
%
Dividing the results among smaller \gls{CSV} files was the first step of the cleaning phase: the next checklist represents the detailed path which
was followed to fulfill the data cleaning.
%
\begin{itemize}
  \item \textbf{Remove "hsl(..., 1, 0.50)"} - It was needed to remove the extra information stored in columns \emph{First Color, Second Color}
   and \emph{Third Color}, since this is redundant because it never varies from entry to entry of the table (remember Section \ref{sec:impl_objectives}).
   These values are the \emph{Saturation} (S) and \emph{Value} (V), primitives of the HSV Color Model used.
  \item \textbf{Format Values} - This step was performed just after the previous one. The value which remains to be formatted is simply the \emph{Hue} (H),
  which is equal to a very precise position on the coded color slider on the interface; the value was composed of 14 decimal numbers, giving us much more
  precision than what is, in fact, needed considering that the hue is measured in terms of integer numbers. The number was rounded up to its closest
  integer number, then. Besides that, there was still one value to be adjusted which was the missing response: \emph{NONE} neeeded to be
  replaced by 0, to simplify the processing of null answers.
  \item \textbf{Sort Entries} - In order to favour the iteration when processing the data, each line of the "Results" Table was
  sorted according, firstly to the \emph{Question ID}, and after by \emph{User ID}.
  \item \textbf{Normalize Laboratory Data} - As previously said, to perform the Laboratory Study we used a Spyder Color Calibrator to manage the color
  representation independently of the environmental conditions of light. Since the Color Profile file generated by the calibrator was used to adapt colors
  to be presented to the user, those same colors had to be trackbacked to the original color, for the sake of normalization of values. This is
  specially useful when comparing the results from this environment to the "Online" Results, helping in data processing later.
  \item \textbf{Verify Duplicated Entries} - This step was performed only to ensure that the entries would not have any matching copy. As expected,
  there were not found any copies.
  \item \textbf{Normalize Profiling Info} - Regarding the "Profiling" data, there was some which was written in Portuguese and other in English, depending on
  the language to perform the study chosen by the user. To avoid misleading profiling categories, all of the academic degrees were normalized to its corresponding
  name both in English and Portuguese. Also, the raw language values contained some specification of English dialects (\emph{e.g.} en\_US, en\_UK) and other languages,
  which was more information than we actually needed; these values were normalized to correspond only to its native and original language (like English, solely).
  \item \textbf{Sanitizing Users} - The tables contained many entries from users that performed the study with incorrect calibration and from users which
  gave unexpected values on the color deficiencies test phase; the entries which corresponded to a user that failed all 6 values on the later phase, would be
  deleted, leaving no trace of its participation. Concerning the bad calibration values, it "opened a window" to investigate the resilience of results when the
  calibration was not what it was expected - this will covered in sub-section \ref{subsec:results_calibration}. To end up the cleaning phase,
  it was decided to treat the color deficient users independently: we separated their values from the regular users to perform a qualitative evaluation.
\end{itemize} \par
%
An example of clean data can be found in table \ref{table:csv_resultsclean}. The next step of data handling is processing it to prepare metrics, establish comparations
to pre-calculated answers and depict results in a CIE Chromaticity Diagram. More tables can be found in Appendix \ref{appendix:tables}, specifically Section
\ref{appendix:sec_results}. \par
%
\begin{table}[htbp]
  \resizebox{\textwidth}{!} {
  \begin{tabular} {|c|c|c|c|c|c|c|c|c|c|}
    \hline
    User ID & Type & First Color & Second Color & Third Color & Drags & Time & Rating & Resets & Question ID \\ \hline
    5713a02a13044 & objTwoColors & \#00FF00 & 0 & 137 & 459 & 56 & 2 & 0 & 17 \\ \hline
    573e4d0eb795b & objTwoColors & \#00FF00 & 235 & 59 & 121 & 28 & 4 & 0 & 17 \\ \hline
    573edae85268b & objTwoColors & \#00FF00 & 242 & 57 & 224 & 20 & 5 & 0 & 17 \\ \hline
    5740ad339507d & objTwoColors & \#00FF00 & 228 & 67 & 205 & 14 & 3 & 0 & 17 \\ \hline
    573c70dabcfe0 & objTwoColors & \#00FF00 & 55 & 221 & 192 & 14 & 2 & 0 & 17 \\ \hline
    57582b17cd76a & twoColorsObj & \#FF0000 & \#00FF00 & \#AF0049 & 724 & 65 & 2 & 0 & 18 \\ \hline
    573c783748e8b & twoColorsObj & \#FF0000 & \#00FF00 & \#BFBE00 & 656 & 47 & 3 & 0 & 18 \\ \hline
    573e4022949b1 & twoColorsObj & \#FF0000 & \#00FF00 & \#B000FF & 334 & 23 & 2 & 0 & 18 \\ \hline
    571151812791a & twoColorsObj & \#FF0000 & \#00FF00 & \#C9B2A2 & 110 & 39 & 2 & 0 & 18 \\
    \hline
  \end{tabular}}
  \caption[Excerpt of Clean "Results" Table]{Excerpt of Results Table, with clean data.}
  \label{table:csv_resultsclean}
\end{table}
%
\section{Data Processing}
\label{sec:results_digest}
%
Processing the data was an important part of the process, since it was important to prepare the raw data collected and compute additional metrics which could be further
analyzed to answer the raised questions. To perform this processing, we decided to implement a set of scripts in \emph{Matlab} which could gauge the dataset of each question,
demographic group and subset of users (non-calibrated and color vision deficients). \par
%
With this data processing, we intend to verify each answer-pair given by a certain user and compare the pairs with each other. It was important to separate the results by question
ID, compare each questions' results with other questions that could conceive the same results, blend the values to check which color model answers are closer to (either HSV, RGB,
CMYK, CIE-L*a*b* or CIE-L*C*h*) and also, give meaning to each value, attributing a name to each color. All these parameters and computations are describred in the next two
sub-sections.
%
\begin{lstlisting}[frame=single]
  % Pseudo-codigo generico;
  % Estrutura do ficheiro;
\end{lstlisting}
%
\subsection{Data Preparation}
\label{subsec:results_preparation}
%
Given the fact that questions had some differences between each other, there would have to be a cautious analysis; to achieve this, we developed a script for each question, each of
file contains the particular set of characteristics ans specific comparisons and values of each question. An exemplary structure of these files can be found on pseudo-code box above.
Each file is capable of computing, for each question, the following datasets:
%
\begin{itemize}
  \item Laboratory Results (Regular Users);
  \item Laboratory Results (Daltonic Users);
  \item Online Results (Regular Users);
  \item Online Results (Daltonic Users);
  \item Online Results (Uncalibrated Users);
  \item Demographic Groups: Users Aged Below 20 Years Results;
  \item Demographic Groups: Users Aged Betweeen 20 and 29 Years Results;
  \item Demographic Groups: Users Aged Betweeen 30 and 39 Years Results;
  \item Demographic Groups: Users Aged Betweeen 40 and 49 Years Results;
  \item Demographic Groups: Users Aged Betweeen 50 and 59 Years Results;
  \item Demographic Groups: Users Aged Above 60 Years Results;
  \item Demographic Groups: Female Users Results;
  \item Demographic Groups: Male Users Results;
  \item Demographic Groups: Other Gender Users Results;
  \item Demographic Groups: White Answers (this computation is only available for Questions 1 to 17).
\end{itemize} \par
%
Dizer o que cada bloco de codigo faz. Falar de diferenças para cada tipo de questão. \par
%
Conversão de respostas com perfil icc novamente, mistura das respostas de acordo com modelos de cor, etc.
Detalhar.
Incluir a mesma tabela que em \ref{subsec:design_core}, com cores, mas com numero de respostas online, lab, e demo.
%
\subsection{Color Bins Comparation}
\label{subsec:results_preparation}
%
De onde apareceram os Bins, como eles são em bruto, incluir esquema do XKCD. Que tratamento foi dado, os problemas
com o desenho dos mesmos e a comparação contra os pontos (em vez da área, que seria o ideal). O que esperavamos
(áreas bem definidas, poligonos bem delineados que daria para desenhar o convexhull), as colisões entre áreas (valores
comuns entre alguns Color Bins) e falar do facto de como se podia, alternativamente, encontrar o nome das cores (diagrama
que já existe desde 1974 - ver ref - mas que não existe um svg ou codigo de todos os pontos, pelo que ainda havia essa curva
de implementação). Poder-se-ia atribuir um nome à cor pela temperatura da mesma, mas não existia uma tabela credível de valores
a utilizar.
%
\section{Results}
\label{sec:results_results}
%
Interpretação de valores. Começar por valores de laboratório. Online serve para corroborar. \par
Incluir tantas tabelas quantas necessárias em cada secção, adaptadas a cada secção e não standardizadas.
%
\subsection{User Profile}
\label{subsec:results_userprofile}
%
\subsection{Color Mixtures}
\label{subsec:results_colormixtures}
%
Fazer também mistura mais fácil, comparando os ratings das questões e ver qual a mistura que apresenta melhores resultados. \\
Comparar misturas que originam a mesma cor, com base em primárias diferentes e perceber se utilizadores conseguem detectar várias
misturas para uma mesma cor.
%
\subsection{Color Models}
\label{subsec:results_colormodels}
%
\subsection{Color Naming}
\label{subsec:results_namingcolors}
%
Cores mais comuns em algumas perguntas; existe alguma ordem caracteristica quando utilizador especifica uma mistura?
%
\subsection{Demographic Groups}
\label{subsec:results_demographic}

\section{Discussion}
\label{sec:results_discussion}

\subsection{Calibration Resiliency}
\label{subsec:results_calibration}
%
Como verificamos ainda alguns users com calibração imprópria para teste, considerámos que poderia ser uma fonte de resultados
interessantes. Como tal, criámos um dataset para os mesmos e comparámos com os resultados dos utilizadores calibrados. Os resultados
são os que se seguem... \par
%
\subsection{Creation of Color Scales}
\label{subsec:results_discussion_colorscales}

\subsection{Color Organization}
\label{subsec:results_discussion_colororganization}

\subsection{Consequences for InfoVis}
\label{subsec:results_discussion_infovis}
%
Resumo dos resultados todos e regras que se podem levar deste trabalho para a área de InfoVis em geral.
%
