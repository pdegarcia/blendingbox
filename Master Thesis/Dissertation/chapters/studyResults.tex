%!TEX root = ../dissertation.tex

\chapter{Research Results}
\label{chapter:results}

\section{Protocol}
\label{sec:results_protocol}

\section{Data Cleaning}
\label{sec:results_datacleaning}

\section{Data Processing}
\label{sec:results_digest}
%
Como foram tratados os dados, no Matlab? Como foram preparados para a análise?
Que processamento for feito aos dados? \\
Incluir a mesma tabela que em \ref{subsec:design_core}, com cores, mas com numero de respostas online, lab, e demo.
%
\subsection{Data Preparation}
\label{subsec:results_preparation}
%
Conversão de respostas com perfil icc novamente, mistura das respostas de acordo com modelos de cor, etc.
Detalhar.
%
\subsection{Color Bins Comparation}
\label{subsec:results_preparation}
%
De onde apareceram os Bins, como eles são em bruto, incluir esquema do XKCD. Que tratamento foi dado, os problemas
com o desenho dos mesmos e a comparação contra os pontos (em vez da área, que seria o ideal). O que esperavamos
(áreas bem definidas, poligonos bem delineados que daria para desenhar o convexhull), as colisões entre áreas (valores
comuns entre alguns Color Bins) e falar do facto de como se podia, alternativamente, encontrar o nome das cores (diagrama
que já existe desde 1974 - ver ref - mas que não existe um svg ou codigo de todos os pontos, pelo que ainda havia essa curva
de implementação). Poder-se-ia atribuir um nome à cor pela temperatura da mesma, mas não existia uma tabela credível de valores
a utilizar.
%
\section{User Profile}
\label{sec:results_userprofile}

\section{Results}
\label{sec:results_results}
%
Interpretação de valores. Começar por valores de laboratório. Online serve para corroborar. \par
Incluir tantas tabelas quantas necessárias em cada secção, adaptadas a cada secção e não standardizadas.
%
\subsection{Color Mixtures}
\label{subsec:results_colormixtures}
%
Fazer também mistura mais fácil, comparando os ratings das questões e ver qual a mistura que apresenta melhores resultados. \\
Comparar misturas que originam a mesma cor, com base em primárias diferentes e perceber se utilizadores conseguem detectar várias
misturas para uma mesma cor.
%
\subsection{Color Models}
\label{subsec:results_colormodels}
%
\subsection{Color Naming}
\label{subsec:results_namingcolors}
%
Cores mais comuns em algumas perguntas; existe alguma ordem caracteristica quando utilizador especifica uma mistura?
%
\subsection{Demographic Groups}
\label{subsec:results_demographic}

\section{Discussion}
\label{sec:results_discussion}

\subsection{Calibration Resiliency}
\label{subsec:results_calibration}
%
Como verificamos ainda alguns users com calibração imprópria para teste, considerámos que poderia ser uma fonte de resultados
interessantes. Como tal, criámos um dataset para os mesmos e comparámos com os resultados dos utilizadores calibrados. Os resultados
são os que se seguem... \par
%
\subsection{Creation of Color Scales}
\label{subsec:results_discussion_colorscales}

\subsection{Color Organization}
\label{subsec:results_discussion_colororganization}

\subsection{Consequences for InfoVis}
\label{subsec:results_discussion_infovis}
%
Resumo dos resultados todos e regras que se podem levar deste trabalho para a área de InfoVis em geral.
%
