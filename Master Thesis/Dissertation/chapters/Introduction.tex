%!TEX root = ../dissertation.tex

\chapter{Introduction}
\label{chapter:introduction}
%
Color can inspire, affect your mood, influence your attitude and change your opinion. It has been associated
to brands, constituting a powerful way of creating instant associations in people’s mind. It is one of the most
interesting subjects of research due to its psychological and physiological complexity and, nowadays, it is
impossible to dissociate what you see around from a color: Starbucks\textsuperscript{\textregistered} has a
strong connection to Green, Target\textsuperscript{\textregistered} to Red, UPS\textsuperscript{\textregistered}
to Brown, Wimbledon Championships\textsuperscript{\textregistered} are strongly associated to Purple and Green
and Facebook\textsuperscript{\textregistered} to Blue. \par
%
Everything around us produces sensations on us, which will be parsed by our sensorial system based on a set
of principles that help our brain build the perceptual world, filled with familiar or non-familiar concepts:
by developing a mental process that represents awareness and knowledge of the real world, it aids the creation
of mental models and improves responsiveness to different \emph{stimuli}. \par
%
As it is stated by Chirimuuta \cite{Chirimuuta2014}, Color is a subjective interpretation of an objective physical
\emph{stimulus}, which may differ from person to person. We, as humans, do not equally perceive color: by saying
this, it is affirmed that the definition and the interpretation of a colored \emph{stimuli} can diverge depending
on the philosophical mindset a person follows. Color has been object of study of different Philosophies and the
definitions of this concept can fluctuate from the simplest statement that colors are simple, primitive intrinsic
properties of physical objects, to the description that colors are subjective properties projected onto object’s
physical surface and light-sources. \par
%
Color perception is influenced by cultural patterns and the environment in which we evolved as a specie; some
tribes in Africa are able to tell more differences between different shades of green that any other color, since
the need to distinguish beneficial from maleficent plants urged and was passed through evolved generations. \par
%
Hence, creating colors standards was more than a need, creating rules for the color usage across different areas
of our society. The formation of color is based on the principle of combining a red, a green and a blue light
source, which will determine the color perceive by the brain. As it is known, the human visual system can only
perceive light from a well defined wavelength range (from under 400 nanometers until 750 nanometers, approximately)
and, consequently, determining the spectrum of colors which is human perceivable. The colors are combined and
represented by Color Models (\emph{e.g.} RGB, HSV or CMYK) which have their own color gamut: the combination
of all these gamuts could be represent in a Color Space like CIE-XYZ, which maps all the human perceivable
colors in a Chromaticity Diagram, frequently named as a Horseshoe Diagram given its shape. \par
%
Color is, nowadays, remarkably used as a powerful tool to convey information: it is used on statistical graphics,
cartographical data, information visualization and developers are eager to use color in their interfaces to
create a better User Experience - when accompanied of an appropriate \emph{Color Scheme}. Particularly, when
showing data variables on a graphic, it is commonly associated to each variable a color and relationships between
them are concluded by observing it. Certainly, it would be useful to combine variables in the same graphic, using
a technique of \emph{Color Blending}, conveying exactly the same information but, from a Computer Graphics
perspective, in an economical way. Nonetheless, there are always bad examples for this technique that yield
terrible visualization results, just by not choosing an appropriate color scale or not taking into account the size
and shape of the subjects to present. \par
%
The technique of \emph{Color Blending} has not been widely exposed and investigated, but some interesting advances
have been made yet. It has been researched if the blending of colors for data visualization \cite{Gama20141} would
be a proper technique to convey information, so as for Visualizing Social Personal Information \cite{Gama20143}.
On the other hand, it is important to understand if users are able to perceive different amounts of blended colors
\cite{Gama20142}, which end up representing different values for data variables: would it be counter-productive to
show data variables in a blended mode, if the users could not tell which variable had the highest value? If not,
how many variables are users capable of distinguish when blended? \par
%
Some researchers have developed parallel techniques which use Color Blending in different
ambits, such as blending colors forming stitched patterns, improving the perception of originally mixed
colors \cite{Urness2003}, or methods to create colors that naturally mix without confusing colors to user \cite{Chuang2009}. \par
%
Even though investigation has been done, there are flaws and situations raised from them which remain to be fully
tested and understood. These led us to state our research goal as: \\ \\
%
\textbf{Study if color blendings can be detected by the users, while testing if it is easier for
users to estimate the pair of colors that resulted in a particular given blend, or reciprocally, to estimate
which blend will result from a given pair of colors.} \\ \\
%
Additionally, we believe that it would be interesting to \textbf{observe if there are any color blendings which
may yield better results than others}, and \textbf{infer if the users follow some kind of mental convention and
organize the color when conveying the answers}, formulating possible implications of color blending usage, in
InfoVis field. \par
%
Having the goals stated, we decided to conduct a user study to fulfill them and answer a small set of objective
questions; this color study had to be realized in a \ul{controlled environment} where all calibration
details were validated to ensure a correct visualization, and also in \ul{an online environment}, since it
was mandatory to gather the largest user sample possible to corroborate the laboratory results. \par
%
We have decided to develop a platform which would be capable of dealing with laboratory and online data: this way,
the overhead of analyzing data from two different environment would be reduced and data was concentrated in the same
place. The study should be divided into four important phases: \textbf{User Profiling Phase, Calibration Test Phase,
Color Vision Deficiencies Test Phase} and, finally, \textbf{Core Test Phase}, which is when we present the color
blendings to the users. These phases are important since full coherence between all user data is a major concern. \par
%
It was created a set of color blendings based on the primitives from the RGB and CMYK Color Model: Red, Green, Blue,
Cyan, Magenta and Yellow. This set was formed by blendings of two and three colors, having the first ones been tested, either
by being given the result of the color blending to the user and asked for the basis, or by giving the basis and asking
for the result which the user though it would be appropriate. These colors were blended following the HSV, RGB, CMYK,
CIE-L*a*b* and CIE-L*C*h* Color Models, interpolating the colors from each pair accordingly. \par
%
In the end of the study, we analyzed the data and divided the results according to: \ul{study environment}, \ul{color
models}, \ul{age groups} and \ul{gender groups}, and \ul{deficient or non-deficient color vision users}. The analysis
of these groups was also supported chromaticity diagrams, which depicted the answers-pairs and reference answers
mapped according to the standard CIE-XYZ. The analysis revealed that the color blending which constantly yielded the
best results across all color models is the \textbf{mixing of Red-Yellow, to achieve Orange}, while the mixture which
provided the worst results when evaluating the distance from the user’s answers and the ideal answers, is the blending
of \textbf{Green-Magenta, resulting in a Blue shade}; these results are consistent with the fact that the human color
perception is conditioned by the amount of Cones present in our eyes. \par
%
When analyzing the answers by Color Model, we can observe that the CMYK Color Model is the one which presents the best
results across both study environments, while the CIE-L*C*h* Color Model is the one which typically provided the worst
results across all color blendings. We have also found that, for color blendings which involved the Red color there was,
in fact, evidence that the users sort the colors when indicating the blending-basis, revealing some mental color organization. \par
%
We have found some research opportunities along the analysis, namely \ul{evaluating the answers which came from
uncalibrated users} because it could show us how resilient the user's expectation is to calibration changes, and \ul{explore
the answer-pairs which contain one white component} which could reveal that either the user did not know how to blend
the colors, or did use the white answer to lighten the color given a bit. \par
%
The focus of this Master thesis was to conclude relevant implications of using color blending techniques, in the Information
Visualization field of research, which are going to be discussed later on this document.
%
\section{Contributions}
%
This dissertation aims to provide useful inputs about how color blending can be correctly used in Information Visualizations,
and according to the users' expectations, leading to the following contributions:
%
\begin{itemize}
  \item \ul{A set of results which determine the answers to the aforementioned questions}, obtained with online and laboratory
  users. These results are organized in tables, divided according to each type of environment and users' condition, and are
  already cleaned and ready to be used.
  \item \ul{A user studies platform called \emph{BlendMe!}}, which is composed of four test phases, helping the process of creating
  a user profile, testing the calibration of an LCD Display, evaluating the presence/absence of color vision deficiencies and collect
  user feedback. Currently, the main core phase of the study is oriented to evaluate 32 questions about color blendings, but it can
  be easily changed.
  \item \ul{A compendium of guidelines on how to use color blending in Information Visualization}, produced based on users'
  results, establishing aspects about color models and blendings which provided the best results.
\end{itemize}
%
\section{Dissertation Outline}
%
The rest of this document is organized as follows: \ul{Chapter 2} provides simultaneously a theoretical background on the
color theory and the research realized on the related work previously accomplished in this research area, \ul{Chapter 3}
presents the design and implementation of the \emph{BlendMe!} platform, \ul{Chapter 4} explores the results' analysis and
diagrams, discussing these results and presenting possible implications for InfoVis. Finally, \ul{Chapter 5} concludes the
entire thesis document, briefly talking about the limitations found, introducing some ideas for future work. \par
%
All the implementation files, analysis scripts and other research-related documents are available in the
public \emph{GitHub} repository: \url{github.com/pdegarcia/blendingbox}. The cleaned results are available in another \emph{GitHub} repository:
\url{github.com/pdegarcia/willitblend-results}.
%
