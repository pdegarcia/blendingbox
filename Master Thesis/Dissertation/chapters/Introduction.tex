%!TEX root = ../dissertation.tex

\chapter{Introduction}
\label{chapter:introduction}
Introduction goes here. \par
%
\textbf{\underline{\emph{... USAR ESTA PARTE DA INTRODUÇÃO DO RESEARCH PROPOSAL ...}}} \par
%
In this section, we introduce the majority of topics to be further studied, the different phases of our research, the metrics we are going to collect and how we are going to treat them. Since we aim to \emph{study to what extent can a user distinguish different amounts of blended colors, when using color mixtures to convey information}, it is important learn from previous results, testing out not only the validity of them but also some missed opportunities. \par
There are several aspects to be considered when developing the broadest study possible: regarding color blending profiling tests, it exists - among
others - some questions which remain unanswered; some of them were risen in the studies by Gama and Gonçalves \cite{Gama20141,Gama20142}. These questions can be divided in four categories:
%
\section{Dissertation Outline}
Describe the organization of the dissertation document, referring to other chapters.
