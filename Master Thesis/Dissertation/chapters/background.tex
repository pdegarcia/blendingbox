%!TEX root = ../dissertation.tex

\chapter{Background}
\label{chapter:background}

\section{Theoretical Background}
\label{sec:theory_background}

\subsection{Color Perception}
\label{subsec:colorperception}
%
In this section, we overview the philosophy about color, relating real-world perceptions to human perception
through the eye. There are two main areas of interest, called the Cones and the Rods, which will be explained
with quite detail; these areas restrict how we perceive color, specially if there are visual deficiencies.
Moreover, color creates mental models and codes, which are part of routines and rules followed by society, and will be exemplified in the end of this section.
%
\subsubsection{Color Philosophy}
\label{sec:colorphilosophy}
Looking up for a concrete definition of color is a hard task: there are quite a few ways to perceive color. Color
raises serious metaphysical questions\footnote{"Stanford Encyclopedia (...)", Available at: \url{stanford.io/1MWp7Zh}. Last accessed on January 8th, 2016.},
concerning the physical and psychological reality of it. Color is an
important feature of subjects: it allows us to recognize objects, locate them, it fires emotions and
behaviors and supports protocols over the world. Probably, the major problem of color has to do with what
we seem to know about it, into what physical properties of objects and materials express about them; David
Hume defended in 1739 \cite{Hume1739} that “(…) Sounds, colors, heat and cold, according to modern philosophy are not
qualities in objects, but perceptions in the mind (…)”, a highly subscribed dogma. This affirmation describes
two important tendencies, the \textbf{elimininativism} and the \textbf{subjectivism}: the first one is the
view that tells physical objects don’t have an inherent color associated to themselves, the last one states
that color is a subjective attribute of objects. Chirimuuta recently argued \cite{Chirimuuta2014} about the different
mindsets one can have regarding color: its main argument is that color is a subjective interpretation of
an objective physical stimuli and, to justify this statement, he settles a contrast between rival theories
of color. Color \textbf{realists} accept that colors are indeed physical properties of objects but instantaneously,
two questions arise from this: \textbf{1)} what really is color and its properties, and \textbf{2)} do objects really possess
those characteristics? With respect to these questions, we can derive even more theories: \textbf{Primitivism}, \textbf{Reductive Physicalism}, \textbf{Dispositionalism}, \textbf{Projectivism}, \textbf{Subjectivism}, among others. \par
%
Chirimuuta \cite{Chirimuuta2014} also compares the \textbf{Realism} against the \textbf{Anti-Realism}, since in the last one, the metaphysical question
“Can we say that objects are actually colored?” is promptly denied: colors do not physically and mentally
exist and nothing is, in fact, colored; it is even said by the anti-realists that classifying color by its features is an  illusion. As the author afirms, his view is close to \textbf{Relationism}, a theory which fills the gap
between the previous mindsets: to fathom color, we have to consider both the perception and the external
\emph{stimuli}, and treat color as the result of this interaction; the task of interpreting color is part of
the mechanism of accessing multiple properties of objects, as shape, composition, \emph{etc}.
Colors are, therefore, relational properties, with respect to perceivers and circumstances of viewing. \par
Regardless of which opinion you support, color has come to be an undeniable point of major interest of studies:
from philosophy, to psychology, computer science or statical analysis, color plays a major role in presenting
numbers, conveying ideas and spreading information. To endorse this usefulness, different color theories were
discovered all along the years and were accompanied by a profound research about the Human eye.
%
\subsection{The Human Eye}
\label{sec:humaneye}
%
The Animal visual system is a direct consequence of evolutionism: it is perfectly adjusted and adapted
to the way of living of every animal. We don’t hunt like wild animals, but our visual system is
prepared to distinguish a wide range of green colors since we evolved as a species surrounded by green
vegetation and knowing what to eat was a matter of life or death. The human visual system is adapted
to do many things, specially detecting sharpness and color with great precision and sensitivity during
the day light and night, although our night vision isn’t quite accurate. \par
%
Light is electromagnetic radiation, but most of this radiation is invisible to the human eye: it can
perceive light from under 400 nanometers until 750 nanometers, as seen on figure \ref{fig:lightspectrum}. \par
When the light strikes an object, depending of the surface’s material, it can either: be wholly or partly
absorbed, reflected or transmitted; what we perceive as being an object is the light reflected of the
surface. The human eye (Figure \ref{fig:eye}), then, decodes light energy into neural activity:
this light reaches the eye through the \textbf{cornea}, crosses the pupil and is refracted by the lens, coming to a final
projection of a sharp image in the back of the eye, the \textbf{retina}. However, this image is
inverted, as the light rays from the top of the object are being project on the bottom of the retina,
so as the light rays coming from the right side of the object, projected in the left side of the
retina. This image is going to be rearranged by the brain.  \par
%
In order to be rearranged, the light is converted in the retina, which contains specialized cells
- photoreceptors - that convert light energy into neural impulses,
which are send to the brain. There are two main types of photoreceptors: \textbf{cones} and
\textbf{rods},
retinal cells that respond to light due to the absorption of photons in their proteins. Cones are concentrated in the \textbf{fovea}, where the light rays
entering the lens are focused. \par
The image is only sent to the brain after the signal generated by the cones and rods is processed also
in the bipolar cells of the retina, where visual information begins to be analyzed. After all, the brain digests the signals sent as we discussed
previously; moreover, as Attneave \cite{Attneave1954} states in its investigation in 1954, a major function of
the human perception mechanism is to strip way some redundant information present in the \emph{stimuli},
in order to encode or describe the incoming information in a more economical form, than the one in which
it impinges on the receptors. Likewise, sensorial events from different sensory systems may create
interdependencies among each other, either in space or in time, or crosscut both; over his life, any
individual acquires notions about “what-goes-with what” and, as the author states \cite{Attneave1954}, we cannot make
predictions about anything, based merely on the present visual field, but also depend on previous - and,
for that, familiar - visual fields. \par
As covered before, there are some specialized and important neurons
that have the crucial task of capturing and transducing photons: they convert electromagnetic radiation
into trigger-signals to be send to the organism.
The photoreceptors can be classified between \textbf{Cones} and \textbf{Rods}. \par
%
\paragraph{Cones}
%
These cells are responsible for acquiring color vision information, at normal-to-high levels of bright
light. They are condensed in the fovea, which is a rod-free zone. By the time of 1990, in a study performed
by Curcio \emph{et al.} \cite{Curcio1990} it was estimated that in the human retina, the total number of cones ranged
between \underline{4.08 to 5.29 million}, Cone cells are not important to light detection, since they are not light
sensitive; however, color perception is completely instrumented by them. They can be seen on Figure \ref{fig:conesrods} as
being the pink colored structures.
These cells are named for their shapes and contain chemicals - the \emph{photospins} - that respond to light: when the
light strikes these chemicals, they break and create a signal which will be transferred to the brain.
There are three kinds of light-sensitive chemicals in cones and they will be providers for the basis
of color vision, creating the distinction between the number of cone types.
%
\begin{itemize}
\item \underline{S-Cones} (Small Wavelength Sensitive), correspond to Blue color perception.
\item \underline{M-Cones} (Medium Wavelength Sensitive), correspond to Green color perception.
\item \underline{L-Cones} (Large Wavelength Sensitive), correspond to Green-Red color perception.
\end{itemize} \par
%
The difference between the signals derived from the three types of cones, allows the brain to perceive a
continuous range of colors. The distribution of the amount of each type can differ.
%
\paragraph{Rods}
%
These photoreceptor cells function in less intense light, when compared to cone cells. They also acquire
their name because of their elongated, cylindrical shape (the white colored shapes in Figure \ref{fig:conesrods});
their location is on the outer edges of the retina, and the number of rods is around \underline{78 to 107 million}. These
cells are much more sensitive than
cones and they are responsible for night vision: in the dark, as your rods have only one type of light-sensitive chemicals (this is why your ability to see gradually increases in the dark). This limitation in the types of rods is the
reason why they cannot discriminate colors, as the cones. On Figure \ref{fig:colorsensitivity}, it is possible to compare
the light absorbance for different wavelenghts, distributed among Cones and Rods. \\ \par
%
All of this color information is, then, sent to the brain where it will be processed and associated to a
mental model. Psychologists tried to explain how the complete color vision works, and formulated some
theories about that.
%
\textbf{\ul{NEXT: subsection Theories of Color Vision, from main.tex}}
%
\subsection{Color Models and Spaces}
\label{subsec:colormodelspaces}

\section{Related Work}
\label{sec:related_work}

\subsection{Color Blending Research and Techniques}
\label{subsec:colorblending}
%
Ver se houve desenvolvimentos na área em 2016. \par
%
\subsection{User Color Studies Online}
\label{subsec:colorstudies}
%
Ler artigos do David Flatla, investigar se existe trabalho feito na área nos últimos meses. \par
Procurar artigos do CHI. \par
%
\section{Discussion}
\label{sec:background_discussion}
%
Discutir aqui possíveis questões que podem ser abordadas. \par
%
\begin{itemize}
	\setlength\itemsep{0.1em}
	\item \textbf{Questions raised Before}
    \begin{itemize}
    	\setlength\itemsep{0.1em}
			\item Will perceived colors correspond to a particular fixed angular value, in the color wheel?
      \item Which is the best formula to blend colors, in each color model? Is it linear interpolation or another?
      \item In the case in which 3 colors are blended, do observers realize all colors at the same time or do they decompose the mixture, firstly in a mixture of two colors and then a blending of a third color?
      \item What is the best way to present color, without influencing color perception?
      \item Does the user \emph{really} understands which colors are involved in a mixing?
		\end{itemize}
  \item \textbf{Perception Questions}
    \begin{itemize}
    	\setlength\itemsep{0.1em}
    	\item Does the order in which colors are mixed, influence mental mixing models? Are there common patterns among mixing orders?
      \item Do shapes and proximity, influence how color is perceived?
      \item Until which extent does background influence the perception of a subject, in particular a blended color?
      \item If color parameters like Saturation, Value or Luminance change in a blending, does it modify color blending perception?
    \end{itemize}
\end{itemize}

\begin{itemize}
    \item \textbf{Information Visualization Questions}
    \begin{itemize}
    	\setlength\itemsep{0.1em}
		\item Do continuous scales yield better results than discrete color scales?
        \item What is the influence of nominal color scales in perception?
        \item What are the results if no color scale is presented to guide the user?
	\end{itemize}
    \item \textbf{Cultural Questions}
    \begin{itemize}
    	\setlength\itemsep{0.1em}
		\item Does the gender really influences how the color is perceived? Is it possible to observe a significant gap between male and female answers?
\item Is it possible to observe significant differences in observation, depending on user's cultural background?
	\end{itemize}
\end{itemize} \par
%
Although there are these questions whose answers remains to be found, only a portion of them will meet their answers, since this is Master Thesis Research Problem. However, there is a set of these questions which was considered crucial and, consequently, had more priority above others: it was this set which was the focus of our studies. We intended to perform \textbf{three studies} and, in the following sections, it is covered the entire proposal for the first study, the conditions in which the study is ideally performed and other important details.
%
