%!TEX root = ../dissertation.tex

\chapter{Conclusion}
\label{chapter:conclusion}
%
This chapter will conclude the dissertation. It will summarize its major contributions and ramifications and discuss directions for future work. \par
%
In this Master thesis, we aimed to understand if color blendings can be detected by the users; but, as we have seen in this document, color has many
different scopes which are not trivial. It was important to conceive a scientifically adjusted research, since this topic aggregates so many different
areas as psychology, physiology, medicine and computer science at the same time. There are many questions which remain unanswered, about the influence
of our cultural background in color perception tasks, how the background of a subject influences its color, how information visualization is influenced
by the usage of color blending, among others previously referred in this document. Since this project aims to achieve the Master Degree in Computer
Science, we only tackled some of these topics: \emph{understand if color blendings can be detected by the users}, \emph{test if it is better to give
the result already mixed, or indicate the blending-basis and the user creates the color mixture}, \emph{to detect if the users follow some kind of
mental convention and organizate the color when conveying the answers}, and formulate possible implications of color blending usage, in Information
Visualization field of research. \par
%
These goals only constituted the user study of our project, which aimed to answer three questions: \textbf{Q1:} Which Color Model meets best the users'
expectations, when blending two colors?, \textbf{Q2:} Is there evidence of a spatial arrangement of colors, when users are indicating possible color
mixtures' results?, and \textbf{Q3:} Are there evidences from differences across demographic groups, such as the age or gender? \par
%
We gathered a generous amount of 259 users, which helped us study which color model yielded the best results. We have found that CMYK is model that
best resembles the users' expectations, while the orange, green and yellow color-blending results are the ones which generate shorter distances to
between the expected colors and the ones indicated by the users. On the other hand, the CIE-L*C*h* Color Model is the one which is farther apart
from the users' mental model of color. \par
%
It was interesting to observe that there is a mild indicator that it exists a difference between the responses from Female and Male users. However,
since our user sample was not complete enough to determine this difference, and not every color blending has presented statistically significant
differences between genders, there is no substantial ground-truth to formulate any a formal research conclusion. \par
%
We have also formulated a set of guidelines which could be followed when using color blending to convey information, which gathered the conclusions
from the all the study results' analysis and summarized it in rule-of-thumbs to follow. \par
%
With this Master Thesis, we have set some implications for the Information Visualization field of research, from the Color Model to use when presenting
color blendings, to what should be asked the user (either the blending-basis or the result of the blending) in order to maximize the success rate of
each visualization, following the color blendings which yield the best results at the same time they are consistent with the human color perception. \par
%
The advent of Information Visualization brings the eagerness of showing beautiful information, in most efficient and fastest way possible to attract
users: color plays a differential role in this task, creating tools to present multiple appealing information at the same time, using Color Blending
Techniques. The results of this thesis determine valid paths to use color blending to transmit information.
%
\section{Limitations}
\label{sec:limitations}
%
We consider that there were some limitations to this Master Thesis. First of all, \ul{the lack of users interested in participating} in our user study was
evident: we had to spend 20 Euros in a voucher card so the users would feel more interested. We think that, with an even larger user sample, the results
of this color study would be majorly evident. \par
%
Another constraint was \ul{the location available} to conduct the user laboratory study: the study had to be realized in rotating locations since there was
no constant, quiet and well-located space for us to fulfill the user sessions. The absence of this factor would cause the divulgation of the laboratory
sessions to be much easier than what, in fact, was. \par
%
Regarding the implementation of the user study, we believe that the study could be implemented with some minor tweaks which have saved us some time: most
of the data cleaning would be avoided if the collected data were saved in a different fashion (no \emph{NONES}, the discarded information was not saved
\emph{a priori}). Moreover, the anaylsis could have been complemented with a statistical analysis in \emph{R}\footnote{"The R Project for Statistical Computing", Available at:
\url{r-project.org}. Last accessed on October 17th, 2016.}, which is a free software environment for statistical computing and graphics, instead of solely
\emph{Matlab} and \emph{SPSS}. Though \emph{Matlab} provided useful tools to process the colors, it has revealed to have a longer-than-expected duration of implementation
specially concerning a user study which comprised 32 different questions, whose individual analysis could not be standardized. On the other hand, \emph{SPSS}
has proved itself to have a steep learning curve, which delayed the statistical analysis. \par
%
Lastly, we believe that this research could have benefited if he had had another disclosure to users: if we had had the opportunity to broadcast this user
study in other countries, we could have attained a sample of users culturally quite different, which could have enhanced the analysis of other cultural
differences.
%
\section{Future Work}
\label{sec:future_work}
%
This field of research has proved itself to have a tremendous potential, whereby there is a large set of questions which remain unanswered; this set is defined
in the Discussion section of this thesis' Background (Section \ref{sec:background_discussion}). Since the size of the user sample is a major concern when
conducting user study like the one conducted by us, and calibration is an unavoidable issue, it would be interesting to \ul{conceive a remote calibration system}
which would be capable of rendering the web page container according to the user's LCD Display calibration. \par
%
Also, it would interesting to explore the \ul{color blendings of 3 colors}, as Gama and Gonçalves did \cite{Gama20141}, but with the goal of definitely
determining if these color blendings have any chance of being used; it should be developed an entire study just to ascertain it. It could be also interesting
to \ul{vary the parameters like Lightness, Luminance, Saturation}, leading the user to explore different shades of a certain Hue: this could be also related to
the aforementioned implication on color scales in Information Visualization. \par
%
As referred in the Results Section (\ref{sec:results_discussion}), there is still room to improve and \ul{detail the comparison between color models which provide
similar good values}: these models are the HSV, RGB and CIE-L*a*b*. Respecting the color blendings, it could be further analyzed and deepen \ul{the relationship of human
color perception with the Blue color}: although it was the one which produced the weakest results of this color study, it still exerts some kind of influence on
mental models of color of our users. \par
%
Comparing the results given with commonly-named colors was an important part of our analysis. Nonethless, the comparison against the XKCD's Color Bins was not seamless:
the generated bins and areas of coverage of each named color were not perfect, so it would be an interesting research topic \ul{to provide a comprehensive study about
the naming of colors}, in a laboratory environment. \par
%
It was mildly observable in our user study the theory that there is a difference in results between gender groups: further investigation could \ul{deepen if there is, in
fact, any plausible difference between genders} or \ul{age groups}. \par
%
Finally, some \ul{color blending alternatives could be explored} along with this technique, for example the Color Weaving, conveying user color studies to ascertain if
there are concrete and valuable alternatives to the blending of colors.
%
