%!TEX root = ../dissertation.tex

\begin{otherlanguage}{english}
\begin{abstract}
\thispagestyle{plain}
\abstractEnglishPageNumber
Using color to convey information is not a recent rule: its usage is further associated to standards, statistics and computer
science. However, color is a subjective aspect of human perception, as it is strongly influenced by cultural background, childhood
learning and possible existent color vision deficiencies. Over the last years, previous investigation has not come to an agreement
about to which extent can color blending techniques be used, in an effective way, to convey information. \\
%
We intended to study if the users can detect the blending-basis when the result of a given color mixture is given, and vice-versa.
We have found that CMYK is the color model that best resembles users’ expectations, while the orange color-blending results
is the one which generate shorter distances between the reference colors and the ones indicated. Whilst, the CIE-L*C*h* Model is the
one which is farther apart from the users’ mental model of color. We have also detected that there is a mild indicator that it exists
a difference between the responses from Female and Male users. All the data collected was developed through a user study, supported by
an online user studies platform called \emph{BlendMe!}. \\
%
As product of this Master thesis, we collected a set relevant implications of using color blending techniques, besides providing a set
of questions apart from the ones answered by the us, which remain unanswered and could turn out as an interesting source of future work.
% Keywords
\begin{flushleft}

\keywords{colors, blending, InfoVis, perception, calibration, bins}

\end{flushleft}

\end{abstract}
\end{otherlanguage}
