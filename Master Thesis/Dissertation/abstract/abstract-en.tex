%!TEX root = ../dissertation.tex

\begin{otherlanguage}{english}
\begin{abstract}
\thispagestyle{plain}
\abstractEnglishPageNumber
Using color to convey information is not a recent rule: its usage is further associated to standards, statistics and computer
science. However, color is a subjective aspect of human perception, as it is strongly influenced by cultural background, childhood
learning and possible existent color vision deficiencies. Over the last years, research has been made to ascertain if color is the ideal
channel to transmit information; particularly, if the blending of two or more colors can convey, in a efficient way, the information
contained in two or more variables, using color blending techniques. \\
%
Nonetheless, previous investigation has not come to an agreement about to which extent can color blending techniques be used, in
an efficient and effective way, to convey information. \\
%
Our goal is to study if the users can detect the blending-basis when the result of a given color mixture is given, and vice-versa; moreover, it
is important to understand which color model has the best matching with the user's blending expectation, and detect colors which
may yield better results than others. \\
%
In order to achieve this, we have developed a user color study performed in an online and laboratory environment. This study was
based on a platform capable of gathering data from both environment, called \emph{BlendMe!}. Among other important details, this
user study was characterized by the creation of a color slider mechanism of presenting the user picked colors, without being
influenced by other unnecessary colors. \\
%
As product of this Master thesis, we intend to conclude relevant implications of using color blending techniques, in the Information
Visualization field of research. Additionally, we will provide a set of questions apart from the ones answered by the us, which
remain unanswered and could be an interesting source of future work.
% Keywords
\begin{flushleft}

\keywords{colors, models, spaces, blending, InfoVis, color perception, user study, calibration, scales, organization, data
visualization, data processing, blending techniques, color vision deficiencies, demographic analysis}

\end{flushleft}

\end{abstract}
\end{otherlanguage}
