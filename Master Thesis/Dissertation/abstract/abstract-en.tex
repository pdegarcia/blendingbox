%!TEX root = ../dissertation.tex

\begin{otherlanguage}{english}
\begin{abstract}
\thispagestyle{plain}
\abstractEnglishPageNumber
Using color to convey information is not a recent rule: its usage is further associated to standards, statistics and computer
science. However, color is a subjective aspect of human perception, as it is strongly influenced by cultural background, childhood
learning and possible existent color vision deficiencies. Over the last years, research has been made to ascertain if color is the ideal
channel to transmit information; particularly, if the blending of two or more colors can convey, in a efficient way, the information
contained in two or more variables, using color blending techniques. \\
%
Nonetheless, previous investigation has not come to an agreement about to which extent can color blending techniques be used, in
an efficient and effective way, to convey information. \\
%
We intended to study if the users can detect the blending-basis when the result of a given color mixture is given, and vice-versa; moreover, it
was important to understand which color model has the best matching with the user's blending expectation, and detect colors which
may yield better results than others. \\
%
We have found that CMYK is model that best resembles the users’ expectations, while the orange, green and yellow color-blending results
are the ones which generate shorter distances to between the reference colors and the ones indicated by the users. On the other hand,
the CIE-L*C*h* Color Model is the one which is farther apart from the users’ mental model of color. We have also detected that there
is a mild indicator that it exists a difference between the responses from Female and Male users, but only in a few color blendings.
All the data collected was developed through a user study with 259 users, from different genders and ages, which was supported by
an online user studies platform called \emph{BlendMe!}. \\
%
As product of this Master thesis, we collected a set relevant implications of using color blending techniques, in the Information
Visualization field of research. Additionally, we provided a set of questions apart from the ones answered by the us, which
remain unanswered and could turn out as an interesting source of future work, and a cleaned dataset containing the data collected
from the user study which could be further analyzed by other researchers.
% Keywords
\begin{flushleft}

\keywords{colors, models, spaces, blending, InfoVis, color perception, user study, calibration, scales, organization, data
visualization, data processing, blending techniques, color vision deficiencies, demographic analysis}

\end{flushleft}

\end{abstract}
\end{otherlanguage}
