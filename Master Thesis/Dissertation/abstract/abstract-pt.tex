%!TEX root = ../dissertation.tex

\begin{otherlanguage}{portuguese}
\begin{abstract}
\thispagestyle{plain}
\abstractPortuguesePageNumber
A utilização de cores para transmitir informação não é prática recente: está fortemente associada a standards e
nomenclaturas de acordo com cores, apresentação de estatísticas e contagens de dados, para além da ciência da computação.
Contudo, a cor é um aspecto consideravelmente subjectivo da percepção humana do meio que nos envolve, já que é
fortemente influenciado pelas raízes culturais, pelas aprendizagens obtidas na infância e, também, por possíveis deficiências
que existam no sistema visual humano. A cor é também frequentemente utilizada para transmitir informação, ao associa-la a variáveis
de dados: commumente, uma cor é atribuida a apenas uma variável mas, não seria interessante atribuir uma mesma cor a duas variáveis
simultaneamente? Ao longo dos últimos anos, a investigação realizada revelou-se inconclusiva em relação
à utilização de técnicas de mistura de cor. \\
%
O objectivo da nossa investigação é estudar se os utilizadores conseguem detectar as bases de uma mistura de cor quando lhe é fornecido
o resultado da mesma, e vice-versa. Concluimos, entre outros temas, que o modelo CMYK é o que melhor retrata as expectativas de mistura
do utilizador, enquanto que o laranja é a cor que origina distâncias mais curtas entre as cores de referência, e
as indicadas pelos utilizadores; por outro lado, o modelo CIE-L*C*h* é o que mais se distancia do modelo mental de cor dos
utilizadores. Detectámos também existir uma ligeira diferença nos resultados indicados pelos utilizadores femininos e masculinos.
Todos os resultados foram recolhidos com base num estudo realizado com utilizadores, suportado por uma plataforma de estudos de utilizador \emph{online} denominada \emph{BlendMe!}. \\
%
Como resultados desta Tese de Mestrado, retirámos algumas implicações relevantes sobre o uso de técnicas de mistura de cor,
na área de Visualização de Informação.

% Keywords
\begin{flushleft}

\palavraschave{cores, misturas, InfoVis, percepção, calibração, \emph{bins}}

\end{flushleft}

\end{abstract}
\end{otherlanguage}
