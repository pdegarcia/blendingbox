%!TEX root = ../dissertation.tex

\begin{otherlanguage}{portuguese}
\begin{abstract}
\thispagestyle{plain}
\abstractPortuguesePageNumber
A utilização de cores para transmitir information não é uma técnica de origem recente: está fortemente associada a standards e
nomenclaturas de acordo com cores, apresentação de estatísticas e contagens de dados, para além da óbvia ciência da computação.
Contudo, a cor é um aspecto consideravelmente subjectivo da percepção e interpretação humana do meio que nos envolve, já que é
fortemente influenciado pelas raízes culturais, pelas aprendizagens obtidas na nossa infância e, também, por possíveis deficiências
que existam no sistema visual humano. Ao longo dos últimos anos, tem sido realizada investigação que incide num melhor conhecimento
sobre se a cor é, de facto, o melhor canal para se transmitir informação; particularmente, se a mistura de duas ou mais cores pode
transmitir de uma forma capaz a informação de duas ou mais variáveis ao mesmo tempo, recorrendo a técnicas de misturas de cores. \\
%
Ainda assim, a investigação realizada até agora revelou-se inconclusiva no que toca à utilização destas mesmas técnicas. Assim, o
objectivo da nossa investigação é estudar se os utilizadores conseguem detectar as bases de uma mistura de cor quando lhe é fornecido
o resultado da mesma, e vice-versa; revela-se, também, importante perceber que modelos de cor oferecem um melhor mapeamento da
expectativa de mistura dos utilizadores, detetando se existem cores e misturas de cores que originam melhores resultados que outras. \\
%
Concluimos, entre outros temas, que o modelo de cor CMYK é o que melhor retrata as expectativas de mistura do utilizador, enquanto que
o laranja, verde e o amarelo são as cores que originam distâncias mais curtas entre as cores de referência para cada mistura citada, e
as indicadas pelos utilizadores. Por outro lado, o modelo de cor CIE-L*C*h* é o que se distancia mais do modelo mental de cor dos
utilizadores. Detectámos também que existe uma ligeira diferença nos resultados indicados pelos utilizadores femininos e masculinos, mas
apenas em algumas misturas de cor. Todos os resultados foram recolhidos com base num estudo realizado com 259 utilizadores de vários
géneros e idades, o qual foi suportado por uma plataforma de estudos de utilizador \emph{online} denominada \emph{BlendMe!}. \\
%
Como resultados desta Tese de Mestrado, retirámos algumas implicações relevantes sobre o uso de técnicas de mistura de cor,
na área de \emph{Information Visualization}. Adicionalmente, chegámos a um conjunto de questões (para além das por nós respondidas)
que ainda não encontraram a sua resposta, e que se podem revelar objecto interessante de investigação no futuro, bem como um
conjunto de dados já tratados que podem ser analisados por outros utilizadores.

% Keywords
\begin{flushleft}

\palavraschave{cores, modelos, espaços, misturas, InfoVis, percepção de cor, estudo de utilizadores, calibração, escalas, organização,
visualização de dados, processamento de dados, técnicas de mistura, deficiências de visão de cor, análise demográfica}

\end{flushleft}

\end{abstract}
\end{otherlanguage}
