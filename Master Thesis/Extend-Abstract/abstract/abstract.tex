%!TEX root = ../article.tex

\begin{abstract}
  Color is frequently used to convey information, while associating it
to data variables: commonly, one color is attributed to only variable but, would it be interesting to attribute one color to two
data variables simultaneously? Over the last years, research has been made to ascertain if the blending
  of two or more colors can convey, in a efficient way, the information contained in two or more variables, using color blending techniques.
  Nonetheless, previous investigation has not come to an agreement. \\
  %
  With our user study, we have found that CMYK is the color model that best resembles the users’ expectations, while the orange, green and yellow color-blending results
  are the ones which generate shorter distances to between the reference colors and the ones indicated by the users. On the other hand,
  the CIE-L*C*h* Color Model is the one which is farther apart from the users’ mental model of color. We have also detected that there
  is a mild indicator that it exists a difference between the responses from Female and Male users, but only in a few color blendings.
  All the data collected was developed through a user study with 259 users, which was supported by an online user studies platform called \emph{BlendMe!}. \\
  %
  We gathered a set relevant implications of using color blending techniques, in the InfoVis field of research, besides providing a set of
  questions which remain unanswered and could turn out as an interesting source of future work, and a cleaned dataset containing the data collected
  from the user study which could be further analyzed by other researchers.
\end{abstract}
