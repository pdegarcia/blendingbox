%!TEX root = ../article.tex

\section{Conclusion}
\label{sec:conclusion}
%
We aimed to answer three questions: \textbf{Q1:} Which Color Model meets best the users' expectations,
when blending two colors?, \textbf{Q2:} Is there evidence of a spatial arrangement of colors, when
users are indicating possible color mixtures' results?, and \textbf{Q3:} Are there evidences from
differences across demographic groups, such as the age or gender? \par
%
We gathered a considerable amount of 259 users, which helped us study which color model yielded the
best results. We have found that CMYK is model that best resembles the users' expectations, while
the orange, green and yellow color-blending results are the ones which generate shorter distances to
between the expected colors and the ones indicated by the users. On the other hand, the CIE-L*C*h*
Color Model is the one which is farther apart from the users' mental model of color. \par
%
It was interesting to observe that there is a mild indicator that it exists a difference between the
responses from Female and Male users. However, since our user sample was not complete enough to
determine this difference, and not every color blending has presented statistically significant
differences between genders, there is no substantial ground-truth to formulate any a formal research conclusion. \par
%
We have also formulated a set of guidelines which could be followed when using color blending to
convey information, which gathered the conclusions from the all the study results' analysis and
summarized it in rule-of-thumbs to follow. \par
%
With this Master Thesis, we have set some implications for the Information Visualization field of
research, from the Color Model to use when presenting color blendings, to what should be asked the
user (either the blending-basis or the result of the blending) in order to maximize the success
rate of each visualization, following the color blendings which yield the best results at the same
time they are consistent with the human color perception. \par
%
We consider that there were some limitations to this Master Thesis, regarding the lack of users
keen to participate in our study, since we believe that the greater the user sample, the better
the results of these user study would have been. Another constraint was the location available
to conduct the user laboratory study, since the study had to be realized in rotating locations
because there was no constant, quiet and well-located space for us to fulfill the user sessions.
The absence of this factor would cause the broadcasting of the laboratory sessions to be much
easier than what, in fact, was. Lastly, we believe that this research could have benefited if
it had had another disclosure to users: if we had had the opportunity to broadcast this user
study in other countries, we could have attained a sample of users culturally quite different,
which could have enhanced the analysis of other cultural differences.
%
\subsection{Future Work}
%
This field of research has proved itself to have a tremendous potential, whereby there is a
large set of questions which remain unanswered. Since the size of the user sample is a major
concern when conducting user study like the one conducted by us, and calibration is an unavoidable
issue, it would be interesting to \ul{conceive a remote calibration system} which would be capable
of rendering the web page container according to the user's LCD Display calibration. \par
%
Comparing the results given with commonly-named colors was an important part of our analysis.
Nonetheless, the comparison against the XKCD's Color Bins was not seamless: the generated bins and
areas of coverage of each named color were not perfect, so it would be an interesting research topic
\ul{to provide a comprehensive study about the naming of colors}, in a laboratory environment. \par
%
Since it was mildly observable in our user study the theory that there is a difference in results
between gender groups, further investigation could \ul{deepen if there is, in fact, any plausible
difference between genders} or \ul{age groups}. \par
%
Finally, respecting the color blendings, it could be further analyzed and deepen \ul{the
relationship of human color perception with the Blue color}: although it was the one which
produced the weakest results of this color study, it still exerts some kind of influence on
mental models of color of our users. \par
