%!TEX root = ../article.tex

\section{Introduction}
\label{sec:introduction}
%
As it is stated by Chirimuuta \cite{Chirimuuta2014}, Color is a subjective interpretation of an objective physical
\emph{stimulus}, which may differ from person to person. We, as humans, do not equally perceive color: by saying
this, it is affirmed that the definition and the interpretation of a colored \emph{stimuli} can diverge depending
on the philosophical mindset a person follows. Color perception can be influenced by cultural patterns and the
environment in which we evolved as a specie.
As it is known, the human visual system can only perceive light from a well defined wavelength range (from under 400
nanometers until 750 nanometers, approximately) and, consequently, determining the spectrum of colors which is human
perceivable. The colors are combined and represented by Color Models (\emph{e.g.} RGB, HSV or CMYK) which have their
own color gamut: the combination of all these gamuts could be represent in a Color Space like CIE-XYZ, which maps all
the human perceivable colors in a Chromaticity Diagram, frequently named as a Horseshoe Diagram given its shape. \par
%
Color is, nowadays, remarkably used as a powerful tool to convey information: it is used on statistical graphics,
cartographical data, information visualization and developers are eager to use color in their interfaces to
create a better User Experience - when accompanied of an appropriate \emph{Color Scheme}. Particularly, when
showing data variables on a graphic, it is commonly associated to each variable a color and relationships between
them are concluded by observing it. Certainly, it would be useful to combine variables in the same graphic, using
a technique of \emph{Color Blending}, conveying exactly the same information but, from a Computer Graphics
perspective, in an economical way. \par
%
This technique has not been widely exposed and investigated, but some interesting advances have been made yet. It
has been researched if the blending of colors for data visualization \cite{Gama20141} would be a proper technique
to convey information, so as for Visualizing Social Personal Information \cite{Gama20143}; on the other hand, it
is also important to understand if users are able to perceive different amounts of blended colors \cite{Gama20142}. \par
%
Even though investigation has been done, there are flaws and situations raised from them which remain to be fully
tested and understood. These led us to state our research goal as: \\ \\
%
\textbf{Study if color blendings can be detected by the users, while testing if it is easier for
users to estimate the pair of colors that resulted in a particular given blend, or reciprocally, to estimate
which blend will result from a given pair of colors.} \\ \\
%
We decided to conduct a user study to fulfill it and answer a small set of objective
questions. We developed a platform which would be capable of dealing with laboratory and online data: this way,
the overhead of analyzing data from two different environment would be reduced and data was concentrated in the same
place. The study should be divided onto four important phases which will be addressed shortly in Section \ref{sec:blendme}. \par
%
It was created a set of color blendings based on the primitives from the RGB and CMYK Color Model: Red, Green, Blue,
Cyan, Magenta and Yellow. This set was formed by blendings of two, which was presented to the user in two forms:
by being given the result of the color blending to the user and asked for the basis, or by giving the basis and asking
for the result which the user though it would be appropriate. These colors were blended following the HSV, RGB, CMYK,
CIE-L*a*b* and CIE-L*C*h* Color Models, interpolating the colors from each pair accordingly. \par
%
In the end of the study, the analysis revealed that the color blending which
constantly yielded the best results across all color models is the \textbf{mixing of Red-Yellow, to achieve Orange}, while
the mixture which provided the worst results when evaluating the distance from the user’s answers and the ideal answers,
is the blending of \textbf{Green-Magenta, resulting in a Blue shade}; these results are consistent with the fact that
the human color perception is conditioned by the amount of Cones present in our eyes. \par
%
When analyzing the answers by Color Model, we can observe that the CMYK Color Model is the one which presents the best
results across both study environments, while the CIE-L*C*h* Color Model is the one which typically provided the worst
results across all color blendings. We have also found that, for color blendings which involved the Red color there was,
in fact, evidence that the users sort the colors when indicating the blending-basis, revealing some mental color organization. \par
%
The focus of this Master thesis was to conclude relevant implications of using color blending techniques, in the Information
Visualization field of research, which are going to be discussed later on this document.
%
\subsection{Contribution}
%
This dissertation aims to provide useful inputs about how color blending can be correctly used in Information Visualizations,
and according to the users' expectations, leading to the following contributions:
%
\begin{itemize}
  \item \ul{A set of results which determine the answers to the aforementioned questions}, obtained with online and laboratory
  users.
  \item \ul{A user studies platform called \emph{BlendMe!}}, which is composed of four test phases, helping the process of creating
  a user profile, testing the calibration and color vision deficiencies and collect user feedback.
  \item \ul{A compendium of guidelines on how to use color blending in Information Visualization}, produced based on users'
  results, establishing aspects about color models and blendings which provide the best results.
\end{itemize}
