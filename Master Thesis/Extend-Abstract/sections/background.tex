%!TEX root = ../article.tex

\section{Background}
\label{sec:background}

\subsection{Color Perception}
%
The Animal visual system is prepared to distinguish a wide range of green colors since we evolved as a species
surrounded by green vegetation and knowing what to eat was a matter of life or death. The human visual system
is adapted to detect sharpness and color with great precision and sensitivity during the day light and night.
Hence, the light which reaches our human visual system is converted in the retina, which contains specialized
cells - photoreceptors - that convert light energy into neural impulses, which are send to the brain. \par
%
There are two main types of photoreceptors: \textbf{cones} and \textbf{rods}, retinal cells that respond to light
due to the absorption of photons in their proteins; cones are concentrated in the fovea. Particularly, the
first ones can be classified in three different types: S (Small), M (Medium) and L (Large), according to the
type of wavelength which they are sensitive to; the absence of one or more type of these cones causes a color
vision deficiency.
%
\subsection{Mental Models of Color}
%
Mixing colors is a constant: learning how colors are disposed in a color wheel happens in our childhood and, in
most of the case, a subtractive color model like the \textbf{RYB} is taught. As Gosset \emph{et al.} state
\cite{Gossett2004}, the usage or learning of subtractive color spaces, in childhood, modifies the mental color
model of each person. Typically, these models are quite different from the \textbf{RGB} model, which may confuse
the observer, since these types of models constitute additive color spaces and are commonly used in visual
displays. \par
%
Mental models of color are created individually by each user, and it can be influenced by many characteristics
from his environment: Color is even perceptually different among different countries, continents, environments
and genders: as it is known\footnote{"Where Man See White (...)", Available at: \url{www.bit.ly/1AMHgcW}. Last
accessed on October 17th, 2016.} \cite{Ginter2011}, women can detect and describe with much more detail color
than men, since the photoreceptors of men take a little longer wavelengths to perceive hue of a color. Also, it
was investigated the cultural influence of each individual in his color perception: researchers have
explored\footnote{\label{itsnoteasy}"It's Not Easy Seeing Green". Available at: \url{www.nyti.ms/1S71yVo}. Last
accessed on October 17th, 2016.} the differences from the western color perception, and tribal color perception;
the researchers presented a circle of squares with different shades of green, to the Himba tribe from Northern Namibia.
Surprisingly, they were able to detect a larger number of shades of green than a western, non-colorblind person,
which may occur because their environment do not manifest as much colors, and they need to detect different shades
to hunt and pick up vegetation and fruits, which traditional western communities do not need to do. \par
%
\subsection{Color Models \& Spaces}
%
Mixing the three primary color light channels to match any color is no longer an oddity and it constitutes
the basics of \emph{Colorimetry}: it is the science used to quantify and describe the human perception of color.
We can describe color as the following equation \cite{Ware2012}: \textbf{$C = sS + mM + lL$}, where C stands for the color
to be matched, \emph{S, M} and \emph{L} are the primary light sources used to create the final color and that are
detected in three types of cones, \emph{s, m} and \emph{l} represent the precise amounts of each primary lights. \par
%
The concepts of \ul{Color Spaces} and \ul{Color Models}
are often confused but, in fact, they do not present the same idea (although they do use similar conceptions). Color models
exist to mathematically conceive a description of color, in which color spaces will be based and present the equivalent
colors, while the latter represents the gamut of colors described accordingly to the primitives of a color model,
containing not only visible colors but also colors that are impossible to represent on physical devices. \par
%
Regarding \ul{Color Models}, there are two types: \textbf{Additive Color Models} use light to display color mixing, mixing
primary colors such as Red, Green and Blue; equally combined and overlapped, they form white light, whereas
\textbf{Subtractive Color Models} mix colors using paint pigments and the result of any blend is a color that tend to be
darker, the more you mix it. Examples of Color Models are RGB (Red-Green-Blue), HSV (Hue-Sauration-Value) and CMYK
(Cyan-Magenta-Yellow-Key). \par
%
\ul{Color Spaces} allow the representation of reproducible colors on a given physical device, while relating the
description of a color model to actual colors, being a three dimensional object that contains all realizable color combinations.
There are three types of them: \textbf{Device-Dependent}, \textbf{Device-Independent} and \textbf{Working Spaces}. The
most important Color Space is the \textbf{CIE 1931} which, nowadays, is derived by \textbf{CIE 1931 XYZ} and \textbf{CIE 1931
RGB}; the first one is the most important, comprising all color sensations which a human can perceive, standing out as a
standard for other color spaces. Another frequently used color space is the \textbf{CIE-L*a*b*}, where two axis are represented
by the a* and b*, being the first one representative of Red-Green and the latter the Blue-Yellow; the L* variable represents the
lightness. This color space derives in a cubic color space representation, which is recognized as the \textbf{CIE-L*C*h*}.
%
\subsection{Related Work}
%
Usually, when it comes to encoding information, color appears as the number one choice, due to the its ease of perception and
familiarity. When representing more than one colored variable at the same time, it would be useful to perceive interrelations
among them and if the users are able to understand which of these entities are related, or blended. \par
%
Gama and Gonçalves started their research \cite{Gama20141} aiming to study to which extent people are able to, given a specific
color resulting from a mixture of two colors, understand the blended color’s origin; besides, they studied which is the color
model that yields the most accurate results: hardware-oriented color models like RGB or color-printing models such as CMYK,
fail to provide a color perception description, unlike HSV. This pitfall is amended by CIE-L*C*h*, by creating a perceptually
uniform scale to lightness. On the other hand, when these researchers have tested the perception of relative amounts of colors
\cite{Gama20142}, they have concluded that users happen to perceive most colors correctly regarding the pair Red-Yellow and,
likewise, colors in both extremes, even for other pairs: “central colors” are generally the most problematic.\par
%
There was some research regarding the usage of color in InfoVis: particularly, about colormaps which are commonly used
in computer science. Zhou and Hansen \cite{Zhou2016} provide a way to classify colormapping techniques into
a taxonomy for readers to quickly identify the appropriate techniques they might use, classifying representative
visualization techniques that explicitly discuss the use of colormaps; these authors gathered the investigation
performed by other researchers in this paper. Additionaly, the authors classify colormap generation as:
\textbf{procedural, user-study based, rule based} and \textbf{data-driven}. One good example of Color Blending
applied with colormaps is the one studied by Stoffel \emph{et al.} \cite{Stoffel2012}, since they introduced a
new technique for visualizing proportions in categorical data; in particular, they combine bipolar colormaps
with an adapted double-rendering of polygons to simultaneously visually represent the first two categories
and the spatial location.
